\documentclass{article}

\usepackage{fancyhdr}
\usepackage{extramarks}
\usepackage{amsmath}
\usepackage{amsthm}
\usepackage{amssymb}
\usepackage{amsfonts}
\usepackage{tikz}
\usepackage[plain]{algorithm}
\usepackage{algpseudocode}
\usepackage[shortlabels]{enumitem}
\usepackage{gensymb}
\usepackage{booktabs}

\usetikzlibrary{automata,positioning}

%
% Basic Document Settings
%

\topmargin=-0.45in
\evensidemargin=0in
\oddsidemargin=0in
\textwidth=6.5in
\textheight=9.0in
\headsep=0.25in

\linespread{1.1}

\pagestyle{fancy}
\lhead{\hmwkAuthorName}
\rhead{\hmwkClass\ (\hmwkClassInstructor): \hmwkTitle}
\lfoot{\lastxmark}
\cfoot{\thepage}

\renewcommand\headrulewidth{0.4pt}
\renewcommand\footrulewidth{0.4pt}

\setlength\parindent{0pt}

%
% Homework Details
%   - Title
%   - Due date
%   - Class
%   - Section/Time
%   - Instructor
%   - Author
%

\newcommand{\hmwkTitle}{Homework\ 6}
\newcommand{\hmwkClass}{MATH 109}
\newcommand{\hmwkClassInstructor}{Professor Rabin}
\newcommand{\hmwkAuthorName}{\textbf{Nalin Bhardwaj}}
\newcommand{\hmwkAuthorID}{A16157819}
\newcommand{\hmwkAuthorEmail}{nibnalin@ucsd.edu}

%
% Title Page
%

\title{
    \vspace{2in}
    \textmd{\textbf{\hmwkClass:\ \hmwkTitle}}\\
    \vspace{0.1in}\large{\textit{\hmwkClassInstructor}}
    \vspace{3in}
}

\author{\hmwkAuthorName \\ \hmwkAuthorID \\ \hmwkAuthorEmail}
\date{}

\renewcommand{\part}[1]{\textbf{\large Part \Alph{partCounter}}\stepcounter{partCounter}\\}

%
% Various Helper Commands
%

% Useful for algorithms
\newcommand{\alg}[1]{\textsc{\bfseries \footnotesize #1}}

% For derivatives
\newcommand{\deriv}[1]{\frac{\mathrm{d}}{\mathrm{d}x} (#1)}

% For partial derivatives
\newcommand{\pderiv}[2]{\frac{\partial}{\partial #1} (#2)}

% Integral dx
\newcommand{\dx}{\mathrm{d}x}

% Alias for the Solution section header
\newcommand{\solution}{\textbf{\large Solution}}

% Probability commands: Expectation, Variance, Covariance, Bias
\newcommand{\E}{\mathrm{E}}
\newcommand{\Var}{\mathrm{Var}}
\newcommand{\Cov}{\mathrm{Cov}}
\newcommand{\Bias}{\mathrm{Bias}}

\begin{document}

\maketitle

\pagebreak

\section{Problem 5.56}

	Let there exist a real number $x$ such that $x^6+x^4+1 = 2x^2 \implies x^6+x^4-2x^2+1 = 0$.

	However, for all $x \in \mathbb{R}$, $x^6 + (x^2-1)^2 = (x^3)^2 + (x^2-1)^2$. Since for all $a \in \mathbb{R}, a^2 \geq 0$, $(x^3)^2 + (x^2-1)^2 \geq 0$.

	Now two cases arise:

	\subsection{$(x^3)^2 + (x^2-1)^2 = 0$}
		However, for $a, b \in \mathbb{R}$, if $a^2+b^2 = 0$, $a$ and $b$ must be $0$. Therefore, $x^3 = 0 \implies x = 0$ and $x^2-1 = 0 \implies x = 1$, which is a contradiction.
	
	\subsection{$(x^3)^2 + (x^2-1)^2 > 0$}
		In this case, $x^6+x^4-2x^2+1 > 0$, therefore, we have a contradiction.
		
	Since we encountered a contradiction in both cases, the statement must be false.

\section{Problem 5.62}

	\begin{enumerate}[(a)]
		\item Let $a^2 + 1 = 2^n$ for even integer $a \geq 2$ and $n \geq 1$.
			
			Since $a$ is even, $a \equiv 0 \mod 2 \implies a^2 \equiv 0 \mod 2 \implies a^2 + 1 \equiv 1 \mod 2$.
			
			However, $2^n = 2*2^{n-1}$, therefore $2^n = a^2+1 \equiv 0 \mod 2$, which is a contradiction.
			
			Therefore, $a$ must be odd.
		\item Let $a^2 + 1 = 2^n$ for integers $a \geq 2$ and $n \geq 1$.
		
			Four cases arise:

			\begin{eqnarray}
				a \equiv 0 \mod 4 \implies a^2 \equiv 0 \mod 4 \implies a^2+1 \equiv 1 \mod 4\\
				a \equiv 1 \mod 4 \implies a^2 \equiv 1 \mod 4 \implies a^2+1 \equiv 2 \mod 4\\
				a \equiv 2 \mod 4 \implies a^2 \equiv 0 \mod 4 \implies a^2+1 \equiv 1 \mod 4\\
				a \equiv 3 \mod 4 \implies a^2 \equiv 1 \mod 4 \implies a^2+1 \equiv 2 \mod 4
			\end{eqnarray}
			
			In all cases, either $a^2+1 \equiv 1 \mod 4$ or $a^2+1 \equiv 2 \mod 4$.

			Since $a \geq 2 \implies a^2+1 \geq 5$, $2^n \geq 5 \implies n \geq 2$.
			
			For $n \geq 2$, $4 | 2^n \implies 2^n = 4k$ for some $k \in \mathbb{Z}$. Therefore, $a^2 + 1 = 4k \implies a^2 + 1 \equiv 0 \mod 4$, which is a contradiction.
	\end{enumerate}
	
\section{Problem 5.65}
	
	When suitor 1 says that they do not know, it must imply that at least one of Suitor 2 and Suitor 3 is wearing a gold crown (since if they were both silver crown wearers, suitor 1 must be wearing gold). Therefore, one of suitor 2 and suitor 3 is wearing gold and the other is possibly wearing silver. When suitor 2 says that they don't know either, it must mean that suitor 1 and suitor 3 are both wearing gold crowns. Clearly, not both of them are wearing silver crowns (same reasoning as suitor 1). If exactly one of them is wearing a silver crown, then suitor 2 must themselves be wearing gold, which he did not assert. Therefore, it must be that all three are wearing gold crowns. And the third suitor does not require any extra information to deduce this.

\section{Problem 6.6}

	\begin{enumerate}[(a)]
		\item The number of cubes in an $n\times n \times n$ cube composed of $n^3$ $1\times1\times1$ cubes.
		\item Since at $1^3 = 1 = \frac{1^2(1+1)^2}{4}$, the formula holds true for $n = 1$.
		
			Assume that for some $k \in \mathbb{Z^+}$, $1^3+2^3 + \cdots + k^3 = \frac{k^2(k+1)^2}{4}$.
			\begin{eqnarray*}
				1^3+2^3 +\cdots+ k^3 + (k+1)^3 & = \frac{k^2(k+1)^2}{4} + (k+1)^3 \\
				& = \frac{k^2(k+1)^2 + 4(k+1)^3}{4} \\
				& = \frac{(k+1)^2(k^2+4k+4)}{4} \\
				& = \frac{(k+1)^2(k+2)^2}{4}
			\end{eqnarray*}
			
		Hence, by the principle of mathematical induction, $1^3+2^3 + \cdots + n^3 = \frac{n^2(n+1)^2}{4}$ for all $n \in \mathbb{Z^+}$.
	\end{enumerate}

\section{Problem 6.10}

	Since $a = \frac{a(1-r)}{1-r}$, as $r \neq 1$, the formula holds true for $n = 1$.

	Assume that for some $k \in \mathbb{Z^+}$, $a+ar+\cdots+ar^{k-1} = \frac{a(1-r^k)}{1-r}$. Then,

	\begin{eqnarray*}
		a+ar+\cdots+ar^{k-1}+ar^k = \frac{a(1-r^k)}{1-r}+ar^k \\
		a+ar+\cdots+ar^{k-1}+ar^k = \frac{a(1-r^k+r^k(1-r))}{1-r} \\
		a+ar+\cdots+ar^{k-1}+ar^k = \frac{a(1-r^{k+1})}{1-r} \\	
	\end{eqnarray*}
	
	Hence, by the principle of mathematical induction, $a+ar+\cdots+ar^{n-1} = \frac{a(1-r^n)}{1-r}$ for all $n \in \mathbb{Z^+}$.
	
\section{Problem 6.12}
	\begin{enumerate}[(a)]
		\item $P(k): 9+13+\cdots+(4k+5) = \frac{4k^2+14k+1}{2}$.
		
		\begin{eqnarray*}
			9+13+\cdots+(4k+5)+(4(k+1)+5) = \frac{4k^2+14k+1}{2}+4(k+1)+5 \\
			9+13+\cdots+(4k+5)+(4(k+1)+5) = \frac{4k^2+14k+1+8k+8+10}{2} \\
			9+13+\cdots+(4k+5)+(4(k+1)+5) = \frac{4(k+1)^2+14(k+1)+1}{2} \\
		\end{eqnarray*}
	
		Therefore, $P(k+1)$ is true. $P(k) \implies P(k+1)$ is true.
		
		\item $P(1): 9 = \frac{19}{2}$ is false. Therefore, $\exists n \in \mathbb{N}, P(n)$ is false.
	\end{enumerate}

\section{Problem 6.18}
	\subsection{Lemma 1: For $k \geq 10$, $k^3 > 3k^2+3k+1$}
		Since $k \geq 10$, $k^3 > 10k^2$, $k^3 > 3k^2 + 7k^2$ and $7k^2 > 7\cdot10\cdot k > 3k + 67k$.

		Since $k \geq 10$, $67k \geq 670 > 1 \implies 67k+3k > 3k+1 \implies 70k > 3k+1 \implies 7k^2 > 3k+1 \implies 7k^2+3k^2 > 3k^2 + 3k + 1 \implies k^3 > 3k^2+3k+1$. \\
		
	Since $2^{10} = 1024 > 1000 = 10^3$, $2^n > n^3$ for $n = 10$.
	
	Assume that for some $k \geq 10$, $2^k > k^3$.
	
	Then, $2^{k+1} = 2^k\cdot2 > 2k^3 \implies 2^k\cdot2 > k^3 + k^3$. By lemma 1, $2^k\cdot2 > k^3 + 3k^2+3k+1 \implies 2^{k+1} > (k+1)^3$.
	
	By the principle of mathematical induction, $2^n > n^3$ for all $n \in \mathbb{Z^+} \geq 10$.

\section{Problem 6.23}
	For $n = 0$, $7 | 3^0 - 2^0 = 7|0$ is true.
	
	Assume that $7 | (3^{2k} - 2^k)$ for $k \in \mathbb{Z^{+}}+\{0\}$.
	
	Then, $3^{2k}-2^k = 7a$ for some $a \in \mathbb{Z}$.
	\begin{eqnarray*}
		3^{2k+2}-2^{k+1} = 3^{2k}\cdot9 - 2^k\cdot2	\\
		3^{2k+2}-2^{k+1} = 2\cdot(3^{2k}-2^{k}) + 7\cdot3^{2k} \\
		3^{2k+2}-2^{k+1} = 2\cdot7a + 7\cdot3^{2k} \\		
		3^{2k+2}-2^{k+1} = 7\cdot(3^{2k}+2a) \\
	\end{eqnarray*}

	Since $2a+3^{2k} \in \mathbb{Z}$, $7 | 3^{2k+2}-2^{k+1}$. Hence, by principle of mathematical induction, $7 | (3^{2n}-2^{n})$ for all $n \in \mathbb{Z^+}+\{0\}$.

\section{Problem 6.24}
	\subsection{Lemma 1: For every real number $x$ and positive integer $k$, $1+(k+1)x+kx^2 \geq 1+(k+1)x$}
	
		For all reals $x$, $x^2 \geq 0 \implies kx^2 \geq 0 \implies 1+(k+1)x+kx^2 \geq 1+(k+1)x$. \\
	
	
	Since $x+1 \geq 1+x$, the formula is true for $n = 1$.
	
	Assume that for $k \in \mathbb{Z^+}$, $(x+1)^k \geq 1+kx$.
	
	Then, since $x+1 \geq 0$, $(x+1)^k(x+1) \geq (1+kx)(x+1)$, by lemma 1, $(x+1)^{k+1} \geq x+1+kx^2+kx \geq 1+(k+1)x \implies (x+1)^{k+1} \geq 1+(k+1)x$.
	
	Therefore, by the principle of mathematical induction, $(x+1)^{n} > 1+nx$ for all $n \in \mathbb{Z^+}$.

\section{Problem 6.34}
	$a_1 = 1, a_2 = 2, a_3 = 4, a_4 = 8 \cdots$
	
	Conjecture: $a_n = 2^{n-1}$
	
	Since $a_1 = 1 = 2^0$, the formula is true for $n = 1$.
	
	Assume that the formula is true for $a_i \forall i \leq k$. Then $a_{k+1} = a_k + 2a_{k-1} = 2^{k-1} + 2\cdot2^{k-2} \implies a_{k+1} = 2\cdot2^{k-1} \implies a_{k+1} = 2^k$.
	
	Hence, by the strong principle of mathematical induction, $a_n = 2^{n-1}$ for all $n \in \mathbb{N}$.

\section{Problem 6.37}
	Since $12 = 3\cdot4+7\cdot0$, $13 = 3\cdot2+7\cdot1$, $14 = 3\cdot0 + 7\cdot2$.
	
	Assume that $i = 7x_i+3y_i$ for some $x_i, y_i \in \mathbb{Z^+}+\{0\}$ for all $12 \leq i \leq k$ for some $k \geq 14$.
	
	Let $k-2 = 3x_{k-2}+7y_{k-2}$ for some $x_{k-2}, y_{k-2} \in \mathbb{Z^+}+\{0\}$. Then $k+1 = (k-2) + 3$. Therefore, $k+1 = 3(x_{k-2}+1)+7y_{k-2}$. Since $x_{k-2}+1$ and $y_{k-2} \in \mathbb{Z^+}+\{0\}$, by the strong principle of mathematical induction, all $n \geq 12$ can be represented as $n = 3a+7b$ for some $a, b \in \mathbb{Z^+}+\{0\}$.
	
\section{Beer?}
	\begin{enumerate}[(a)]
		\item If any of the first two mathematicians did not want to drink beer, they would have answered no. However, they did want to drink beer, but they did not know if the other people following them did. Since the third mathematician knew the other two did want to drink beer since they did not answer no, she says "Yes".
		\item No. This is because the probability of this event happening (all 3 mathematicians drinking beer) is $\frac{1}{2^3} = \frac{1}{8}$(each mathematician either drinks beer, or does not), diminishing the funniness of the joke by a factor of 8.
	\end{enumerate}

\end{document}
