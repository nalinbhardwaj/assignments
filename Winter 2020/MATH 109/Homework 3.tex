\documentclass{article}

\usepackage{fancyhdr}
\usepackage{extramarks}
\usepackage{amsmath}
\usepackage{amsthm}
\usepackage{amsfonts}
\usepackage{tikz}
\usepackage[plain]{algorithm}
\usepackage{algpseudocode}
\usepackage[shortlabels]{enumitem}
\usepackage{gensymb}
\usepackage{booktabs}

\usetikzlibrary{automata,positioning}

%
% Basic Document Settings
%

\topmargin=-0.45in
\evensidemargin=0in
\oddsidemargin=0in
\textwidth=6.5in
\textheight=9.0in
\headsep=0.25in

\linespread{1.1}

\pagestyle{fancy}
\lhead{\hmwkAuthorName}
\rhead{\hmwkClass\ (\hmwkClassInstructor): \hmwkTitle}
\lfoot{\lastxmark}
\cfoot{\thepage}

\renewcommand\headrulewidth{0.4pt}
\renewcommand\footrulewidth{0.4pt}

\setlength\parindent{0pt}

%
% Homework Details
%   - Title
%   - Due date
%   - Class
%   - Section/Time
%   - Instructor
%   - Author
%

\newcommand{\hmwkTitle}{Homework\ 3}
\newcommand{\hmwkClass}{MATH 109}
\newcommand{\hmwkClassInstructor}{Professor Rabin}
\newcommand{\hmwkAuthorName}{\textbf{Nalin Bhardwaj}}
\newcommand{\hmwkAuthorID}{A16157819}
\newcommand{\hmwkAuthorEmail}{nibnalin@ucsd.edu}

%
% Title Page
%

\title{
    \vspace{2in}
    \textmd{\textbf{\hmwkClass:\ \hmwkTitle}}\\
    \vspace{0.1in}\large{\textit{\hmwkClassInstructor}}
    \vspace{3in}
}

\author{\hmwkAuthorName \\ \hmwkAuthorID \\ \hmwkAuthorEmail}
\date{}

\renewcommand{\part}[1]{\textbf{\large Part \Alph{partCounter}}\stepcounter{partCounter}\\}

%
% Various Helper Commands
%

% Useful for algorithms
\newcommand{\alg}[1]{\textsc{\bfseries \footnotesize #1}}

% For derivatives
\newcommand{\deriv}[1]{\frac{\mathrm{d}}{\mathrm{d}x} (#1)}

% For partial derivatives
\newcommand{\pderiv}[2]{\frac{\partial}{\partial #1} (#2)}

% Integral dx
\newcommand{\dx}{\mathrm{d}x}

% Alias for the Solution section header
\newcommand{\solution}{\textbf{\large Solution}}

% Probability commands: Expectation, Variance, Covariance, Bias
\newcommand{\E}{\mathrm{E}}
\newcommand{\Var}{\mathrm{Var}}
\newcommand{\Cov}{\mathrm{Cov}}
\newcommand{\Bias}{\mathrm{Bias}}

\begin{document}

\maketitle

\pagebreak

\section*{Problem 2.70}
	\begin{enumerate}[(a)]
		\item There exists rational number $r$, such that the number $\frac{1}{r}$ is not rational.
		\item For every rational number $r$, $r^2 \neq 2$.
	\end{enumerate}

\section*{Problem 2.72}
	\begin{enumerate}[(a)]
		\item True, for $x = 0$, $0^2 - 0 = 0$.
		\item True, since $\forall n \in \mathbb{N} \quad n \geq 1 \implies n+1 \geq 2$.
		\item False, since for $x = -1$, $\sqrt{(-1)^2} = 1 \neq -1$.
		\item True, for $x = 3$, $3\cdot(3)^2 - 27 = 0$ is true.
		\item True, for $x = 5, y = 0$, $x+y+3 = 8$ is true.
		\item False, for $x = y = 0$, $x+y+3 = 8$ is false.
		\item True, for $x = 3, y = 0$, $x^2+y^2 = 9$ is true.
		\item False, for $x = y = 1$, $x^2+y^2 = 9$ is false.
	\end{enumerate}

\section*{Problem 2.93a}
	\begin{tabular}{|l|l|l|l|l|l|l|l|l|}
	\hline
	$P$ & $Q$ & $R$ & $(P \wedge Q)$ & $(P \wedge Q) \implies R$ & $\sim R$ & $P \wedge \sim R$ & $\sim Q$ & $P \wedge (\sim R) \implies (\sim Q)$ \\ \hline
	T & T & T & T                           & T                                                     & F       & F                               & F       & T                                                                   \\ \hline
	T & T & F & T                           & F                                                     & T       & T                               & F       & F                                                                   \\ \hline
	T & F & T & F                           & T                                                     & F       & F                               & T       & T                                                                   \\ \hline
	T & F & F & F                           & T                                                     & T       & T                               & T       & T                                                                   \\ \hline
	F & T & T & F                           & T                                                     & F       & F                               & F       & T                                                                   \\ \hline
	F & T & F & F                           & T                                                     & T       & F                               & F       & T                                                                   \\ \hline
	F & F & T & F                           & T                                                     & F       & F                               & T       & T                                                                   \\ \hline
	F & F & F & F                           & T                                                     & T       & F                               & T       & T                                                                   \\ \hline
	\end{tabular}

	Since the truth values of $(P \wedge Q) \implies R$ and $P \wedge (\sim R) \implies (\sim Q)$ are the same for all possible truth values of $P, R$ and $Q$, they are logically equivalent.

\section*{Problem 2.103}
	\begin{enumerate}[(a)]
		\item The real number $r$ either has the property $r < 3$ or $r \geq \pi$.
		\item There exists integer $n$ such that $|r - n| < \frac{1}{2}$ for the real number $r$.
		\item There exists real number $s$ such that $r \cdot s \neq s$ for the real number $r$.
	\end{enumerate}

\section*{Problem 3.2}
	To prove: If $|n - 1| + |n + 1| \leq 1$, then $|n^2 - 1| \leq 4$. \\ \\
	
	Proof: Let $n \in \mathbb{N}$, since $n \geq 1 \implies n - 1 \geq 0 \implies |n - 1| \geq 0$. \\
	Similarly,  $n \geq 1 \implies n + 1 \geq 2 \implies |n + 1| \geq 2$, \\
	Adding, $|n - 1| + |n + 1| \geq 0 + 2 \implies |n - 1| + |n + 1| \geq 2$. \\
	Therefore, $|n - 1| + |n + 1| \leq 1$ is false $\forall n \in \mathbb{N}$ and the result follows vacuously.

\section*{Problem 3.10}
	To prove: If $a$ and $c$ are odd integers, then $a \cdot b + b \cdot c$ is even for every integer $b$. \\

	Proof: Since $a$ and $c$ are odd, let $a = 2 \cdot k_1 + 1$ and $c = 2 \cdot k_2 + 1$ for some $k_1, k_2 \in \mathbb{Z}$. \\
	\begin{eqnarray*}
		a \cdot b + b \cdot c & = & (a + c) \cdot b \\
		& = & (2 \cdot k_1 + 1 + 2 \cdot k_2 + 1) \cdot b \\
		& = & 2 \cdot ((k_1 + k_2 + 1) \cdot b)
	\end{eqnarray*}
	Since $(k_1 + k_2 + 1) \cdot b \in \mathbb{Z}$, $a \cdot b + b \cdot c$ must be even. 
	
\section*{Problem 3.20}
	To prove: For $x \in \mathbb{Z}$, $3x + 1$ is even if and only if $5x - 2$ is odd.

	\subsection*{Lemma 1: If $3x + 1$ is even, then $x$ is odd.}
		Proof by contrapositive: Let $x$ be an even integer, therefore $x = 2k$ for some $k \in \mathbb{Z}$. \\
		Then, $3x+1$ = $2(3k) + 1$. \\
		Since $3k \in \mathbb{Z}$, $3x+1$ is odd.
	
	\subsection*{Theorem 1: If $3x + 1$ is even, then $5x-2$ is odd.}
		Using Lemma 1, $x$ must be odd. Therefore, let $x = 2k + 1$ for some $k \in \mathbb{Z}$. \\
		Then, $5x-2 = 5(2k + 1) - 2 = 2 \cdot 5k + 3 = 2(5k + 1) + 1$. \\
		Since $5k+1 \in \mathbb{Z}$, $5x - 2$ is odd.

	\subsection*{Lemma 2: If $5x - 2$ is odd, then $x$ is odd.}
		Proof by contrapositive: Let $x$ be an even integer, $x = 2k$ for some $k \in \mathbb{Z}$. \\
		Then, $5\cdot2k - 2 = 2(5k - 1)$. \\
		Since $5k - 1 \in \mathbb{Z}$, $5x - 2$ is even.

	\subsection*{Theorem 2: If $5x - 2$ is odd, then $3x+1$ is even.}
		Using Lemma 2, $x$ must be odd. Therefore, let $x = 2k + 1$ for some $k \in \mathbb{Z}$. \\
		Then, $3x + 1 = 3(2k+1) + 1 = 2(3k+2)$. \\
		Since $3k + 2 \in \mathbb{Z}$, $3x-1$ is even.

\section*{Problem 3.31}

	Contrapositive: If $a$ is odd or $b$ is odd, then $a + b$ and $ab$ are of different parity.
	
	\subsection*{Case 1}
		WLOG, Let $a$ be odd and $b$ be even.
		Therefore $a = 2k_1 + 1$ and $b = 2k_2$ for some $k_1,k_2 \in \mathbb{Z}$. \\
		Then, $a + b = 2k_1 + 1 + 2k_2 = 2(k_1+k_2) + 1$. Since $k_1+k_2 \in \mathbb{Z}$, $a + b$ is odd. \\
	 	$ab = (2k_1+1)\cdot(2k_2) = 2\cdot((2k_1 + 1) \cdot k_2)$. Since $((2k_1 + 1) \cdot k_2) \in \mathbb{Z}$, $ab$ is even. \\
	 	Therefore, $a+b$ and $ab$ have opposite parity. \\
	 
	\subsection*{Case 2}
		Let $a$ be odd and $b$ be odd.
		Therefore $a = 2k_1 + 1$ and $b = 2k_2 + 1$ for some $k_1,k_2 \in \mathbb{Z}$. \\
	 	Then, $a + b = 2k_1 + 1 + 2k_2 + 1 = 2(k_1+k_2+1)$. Since $k_1+k_2 + 1 \in \mathbb{Z}$, $a + b$ is even. \\
	 	$ab = (2k_1+1)\cdot(2k_2 + 1) = 4k_1k_2 + 2(k_1+k_2) + 1 = 2(2k_1k_2 + k_1 + k_2) + 1$. Since $2k_1k_2 + k_1 + k_2 \in \mathbb{Z}$, $ab$ is odd. \\
	 	Therefore, $a+b$ and $ab$ have opposite parity.

\section*{Problem 3.36}

	Contrapositive: If $x$ is odd or $y$ is odd, then $3x+4y$ is odd or $4x+5y$ is odd.
	
	\subsection*{Case 1}
		Let $x$ and $y$ be odd. Therefore, $x = 2k_1 + 1$ and $y = 2k_2 + 1$ for some $k_1, k_2 \in \mathbb{Z}$. \\
		$3x+4y = 3\cdot 2k_1 + 3 + 4\cdot 2k_2 + 4 = 2(3k_1+4k_2+3) + 1$.
		Since $3k_1+4k_2+3 \in \mathbb{Z}$, $3x+4y$ is an odd integer and the implication is true.
	
	\subsection*{Case 2}
		Let $x$ be odd and $y$ be even. Therefore, $x = 2k_1 + 1$ and $y = 2k_2$ for some $k_1, k_2 \in \mathbb{Z}$. \\
		$3x+4y = 3\cdot 2k_1 + 3 + 4\cdot 2k_2 = 2(3k_1+4k_2+1) + 1$.
		Since $3k_1+4k_2+1 \in \mathbb{Z}$, $3x+4y$ is an odd integer and the implication is true.
	
	\subsection*{Case 3}
		Let $x$ be even and $y$ be odd. Therefore, $x = 2k_1$ and $y = 2k_2 + 1$ for some $k_1, k_2 \in \mathbb{Z}$. \\
		$4x + 5y = 4 \cdot 2k_1 + 5 \cdot 2k_2 + 5 = 2(4k_1+5k_2+2) + 1$.
		Since $4k_1+5k_2+2 \in \mathbb{Z}$, $4x + 5y$ is an odd integer and the implication is true.
		
\section*{Problem 3.42}
	The first part proves the implication: If $x$ is even, then $3x^2 - 4x - 5$ is odd. \\
	The second part proves: If $x$ is odd, then $3x^2 - 4x - 5$ is even. By contrapositive, this is equivalent to: If $3x^2 - 4x - 5$ is odd, $x$ is even, which is the converse of the first result. \\
	
	Therefore, overall result is: $x$ is even if and only if $3x^2 - 4x - 5$ is odd.
	
\section*{Extra Problem}
	To prove: Every integer is either even or odd, but not both. \\
	
	By the fundamental theorem of arithmetic, for division by $2$ of any number, remainder on division must be $0$ or $1$. Therefore, by definition, every integer can be expressed as $2k$ or $2k+1$ for some $k \in \mathbb{Z}$.
	
	Now, let us prove that no integer can be both even and odd at the same time.
	
		Let an integer $x$ be both even and odd. Therefore, $x = 2k_1$ and $x = 2k_2 + 1$ for some $k_1,k_2 \in \mathbb{Z}$. Therefore,
		\begin{eqnarray*}
			2k_1 & = & 2k_2 + 1 \\
			2 \cdot (k_1 - k_2) & = & 1 \\
			k_1 - k_2 & = & \frac{1}{2}	
		\end{eqnarray*}
		
		However, subtraction is closed in $\mathbb{Z}$. Therefore, our assumption must be false, and it is impossible for an integer to be even and odd simultaneously.

\end{document}
