\documentclass{article}

\usepackage{fancyhdr}
\usepackage{extramarks}
\usepackage{amsmath}
\usepackage{amsthm}
\usepackage{amssymb}
\usepackage{amsfonts}
\usepackage{tikz}
\usepackage[plain]{algorithm}
\usepackage{algpseudocode}
\usepackage[shortlabels]{enumitem}
\usepackage{gensymb}
\usepackage{booktabs}

\usetikzlibrary{automata,positioning}

%
% Basic Document Settings
%

\topmargin=-0.45in
\evensidemargin=0in
\oddsidemargin=0in
\textwidth=6.5in
\textheight=9.0in
\headsep=0.25in

\linespread{1.1}

\pagestyle{fancy}
\lhead{\hmwkAuthorName}
\rhead{\hmwkClass\ (\hmwkClassInstructor): \hmwkTitle}
\lfoot{\lastxmark}
\cfoot{\thepage}

\renewcommand\headrulewidth{0.4pt}
\renewcommand\footrulewidth{0.4pt}

\setlength\parindent{0pt}

%
% Homework Details
%   - Title
%   - Due date
%   - Class
%   - Section/Time
%   - Instructor
%   - Author
%

\newcommand{\hmwkTitle}{Homework\ 3}
\newcommand{\hmwkClass}{MATH 109}
\newcommand{\hmwkClassInstructor}{Professor Rabin}
\newcommand{\hmwkAuthorName}{\textbf{Nalin Bhardwaj}}
\newcommand{\hmwkAuthorID}{A16157819}
\newcommand{\hmwkAuthorEmail}{nibnalin@ucsd.edu}

%
% Title Page
%

\title{
    \vspace{2in}
    \textmd{\textbf{\hmwkClass:\ \hmwkTitle}}\\
    \vspace{0.1in}\large{\textit{\hmwkClassInstructor}}
    \vspace{3in}
}

\author{\hmwkAuthorName \\ \hmwkAuthorID \\ \hmwkAuthorEmail}
\date{}

\renewcommand{\part}[1]{\textbf{\large Part \Alph{partCounter}}\stepcounter{partCounter}\\}

%
% Various Helper Commands
%

% Useful for algorithms
\newcommand{\alg}[1]{\textsc{\bfseries \footnotesize #1}}

% For derivatives
\newcommand{\deriv}[1]{\frac{\mathrm{d}}{\mathrm{d}x} (#1)}

% For partial derivatives
\newcommand{\pderiv}[2]{\frac{\partial}{\partial #1} (#2)}

% Integral dx
\newcommand{\dx}{\mathrm{d}x}

% Alias for the Solution section header
\newcommand{\solution}{\textbf{\large Solution}}

% Probability commands: Expectation, Variance, Covariance, Bias
\newcommand{\E}{\mathrm{E}}
\newcommand{\Var}{\mathrm{Var}}
\newcommand{\Cov}{\mathrm{Cov}}
\newcommand{\Bias}{\mathrm{Bias}}

\begin{document}

\maketitle

\pagebreak

\section*{Problem 3.64}
	To prove: For $a, b \in \mathbb{Z}$, if $ab = 4$, then $(a-b)^3 - 9(a-b) = 0$

	\subsection*{Lemma 1: If $a-b = 0$ or $(a-b)^2-9 = 0$, then $(a-b)^3-9(a-b) = 0$}
		If $a-b = 0$ or $(a-b)^2-9 = 0$, then $(a-b)((a-b)^2-9) = 0$.
		
		Then, $(a-b)^3 - 9(a-b) = 0$.
	
	\subsection*{Proof}
		If $a, b \in \mathbb{Z}$ and $ab = 4$, then the following cases arise:
	\begin{enumerate}
		\item $a = 1, b = 4$: $(1-4)^2-9 = 0 \implies (a-b)^2 - 9 = 0$
		\item $a = 2, b = 2$: $(2 - 2) = 0 \implies a-b = 0$
		\item $a = 4, b = 1$: $(4-1)^2-9 = 0 \implies (a-b)^2 - 9 = 0$
		\item $a = -1, b = -4$: $(-1+4)^2-9 = 0 \implies (a-b)^2 - 9 = 0$
		\item $a = -2, b = -2$: $(-2 + 2) = 0 \implies a-b = 0$
		\item $a = -4, b = -1$: $(4-1)^2-9 = 0 \implies (a-b)^2 - 9 = 0$
	\end{enumerate}
	
	In all cases, either $a-b = 0$ or $(a-b)^2-9 = 0$. Therefore, using Lemma 1, $(a-b)^3-9(a-b) = 0$.

\section*{Problem 3.68}
	To prove: For $n \in \mathbb{N}$, If $n^3 - 5n - 10 > 0$, then $n \geq 3$.
	
	Contrapositive: If $n < 3$, then $n^3 - 5n - 10 \leq 0$.
	
	Since $n \in \mathbb{N}$, $1 \leq n \leq 2$.
	
	Therefore, $n^3 \leq 8$ and $3 \leq n+2 \leq 4 \implies 15 \leq 5(n+2) \leq 20$.
	
	Therefore, $n^3 \leq 8 \leq 15 \leq 5(n+2) \implies n^3 \leq 5(n+2) \implies n^3-5(n+2) \leq 0 \implies n^3 - 5n - 10 \leq 0$.
	
\section*{Problem 3.74}
	To prove: For $a, b, c \in \mathbb{Z}$, if $a^2+b^2 = c^2$, then $abc$ is even.
	
	\subsection*{Lemma 1: If $x^2$ is even, then $x$ is even}
		Contrapositive: If $x$ is odd, then $x^2$ is odd.
		
		Let $x = 2k+1$ for some $k \in \mathbb{Z}$.
		
		$x^2 = 4k^2+4k+1 = 2(2k^2+2k) + 1$. Since $2k^2+2k \in \mathbb{Z}$, $x^2$ is odd.
	
	
	\subsection*{Case 1: $a$ or $b$ is even.}
		WLOG, let $a$ be even.
		
		Then, $a = 2k$ for some $k \in \mathbb{Z}$.
		
		$abc = 2kbc = 2(kbc)$. Since $kbc \in \mathbb{Z}$, $abc$ is even.

	\subsection*{Case 2: $a$ and $b$ are odd.}
		Let $a = 2k_1+1$ and $b = 2k_2+1$ for some $k_1, k_2 \in \mathbb{Z}$.
		
		Then, $c^2 = a^2+b^2 = 4k_1^2+1+4k_1+4k_2^2+1+4k_2 = 2(2k_1^2+1+2k_1+2k_2^2+2k_2)$.
		
		Since $2k_1^2+1+2k_1+2k_2^2+2k_2 \in \mathbb{Z}$, $c^2$ is even.
		
		Since $c^2$ is even, using lemma 1, $c$ is even.
		
		Let $c = 2k$ for $k \in \mathbb{Z}$.
		
		$abc = ab2k = 2(abk)$. Since $abk \in \mathbb{Z}$, $abc$ is even.
		
\section*{Problem 4.5}
	To prove: For $a, b, c \in \mathbb{Z}$ such that $a \neq 0$, if $a \nmid b$, then $a \nmid b$ and $a \nmid c$.
	
	Contrapositive: If $a \mid b$ or $a \mid c$, then $a \mid bc$.
	
	WLOG, let $a \mid b$.
	
	Then, $b = an$ for some $n \in \mathbb{Z}$.
		
	$bc = anc = a(nc)$. Since $nc \in \mathbb{Z}$, $a \mid bc$.
	
\section*{Problem 4.10}
	\subsection*{Lemma 1: If $2 \mid n^4 - 3$, then $2 \nmid n$}
		Contrapositive: If $2 \mid n$, then $2 \nmid n^4 - 3$.
		
		Let $n = 2k$ for $k \in \mathbb{Z}$.
		
		$n^4 - 3 = 16k^4 - 3 = 16k^4-4+1 = 2(8k^4-2)+1$. Since $8k^4-2 \in \mathbb{Z}$, $2 \nmid n^4 - 3$.

	\subsection*{Lemma 2: If $4 \mid n^2 + 3$, then $2 \nmid n$}
		Contrapositive: If $2 \mid n$, then $4 \nmid n^2 + 3$.
		
		Let $n = 2k$ for $k \in \mathbb{Z}$.
		
		$n^2 + 3 = 4k^2+3$. Since $k \in \mathbb{Z}$, $4 \nmid n^2+3$.
	
	\subsection*{Proof}
		By lemma 1, if $2 \mid n^4-3$, then $2 \nmid n$.
	
		Then, $n = 2k+1$ for $k \in \mathbb{Z}$.
		$n^2 + 3 = 4(k^2+k+1)$. Since $k^2+k+1 \in \mathbb{Z}$, $4 \mid n^2+3$. 

		By lemma 2, if $4 \mid n^2 + 3$, then $2 \nmid n$
	
		Then, $n = 2k+1$ for $k \in \mathbb{Z}$.
		$n^4 - 3 = 16 k^4 + 32 k^3 + 24 k^2 + 8 k + 2 = 2(8 k^4 + 16 k^3 + 12 k^2 + 4 k + 1)$. Since $8 k^4 + 16 k^3 + 12 k^2 + 4 k + 1 \in \mathbb{Z}$, $2 \mid n^4 - 3$.
'



\end{document}
