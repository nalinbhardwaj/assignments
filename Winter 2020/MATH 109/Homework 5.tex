\documentclass{article}

\usepackage{fancyhdr}
\usepackage{extramarks}
\usepackage{amsmath}
\usepackage{amsthm}
\usepackage{amssymb}
\usepackage{amsfonts}
\usepackage{tikz}
\usepackage[plain]{algorithm}
\usepackage{algpseudocode}
\usepackage[shortlabels]{enumitem}
\usepackage{gensymb}
\usepackage{booktabs}

\usetikzlibrary{automata,positioning}

%
% Basic Document Settings
%

\topmargin=-0.45in
\evensidemargin=0in
\oddsidemargin=0in
\textwidth=6.5in
\textheight=9.0in
\headsep=0.25in

\linespread{1.1}

\pagestyle{fancy}
\lhead{\hmwkAuthorName}
\rhead{\hmwkClass\ (\hmwkClassInstructor): \hmwkTitle}
\lfoot{\lastxmark}
\cfoot{\thepage}

\renewcommand\headrulewidth{0.4pt}
\renewcommand\footrulewidth{0.4pt}

\setlength\parindent{0pt}

%
% Homework Details
%   - Title
%   - Due date
%   - Class
%   - Section/Time
%   - Instructor
%   - Author
%

\newcommand{\hmwkTitle}{Homework\ 5}
\newcommand{\hmwkClass}{MATH 109}
\newcommand{\hmwkClassInstructor}{Professor Rabin}
\newcommand{\hmwkAuthorName}{\textbf{Nalin Bhardwaj}}
\newcommand{\hmwkAuthorID}{A16157819}
\newcommand{\hmwkAuthorEmail}{nibnalin@ucsd.edu}

%
% Title Page
%

\title{
    \vspace{2in}
    \textmd{\textbf{\hmwkClass:\ \hmwkTitle}}\\
    \vspace{0.1in}\large{\textit{\hmwkClassInstructor}}
    \vspace{3in}
}

\author{\hmwkAuthorName \\ \hmwkAuthorID \\ \hmwkAuthorEmail}
\date{}

\renewcommand{\part}[1]{\textbf{\large Part \Alph{partCounter}}\stepcounter{partCounter}\\}

%
% Various Helper Commands
%

% Useful for algorithms
\newcommand{\alg}[1]{\textsc{\bfseries \footnotesize #1}}

% For derivatives
\newcommand{\deriv}[1]{\frac{\mathrm{d}}{\mathrm{d}x} (#1)}

% For partial derivatives
\newcommand{\pderiv}[2]{\frac{\partial}{\partial #1} (#2)}

% Integral dx
\newcommand{\dx}{\mathrm{d}x}

% Alias for the Solution section header
\newcommand{\solution}{\textbf{\large Solution}}

% Probability commands: Expectation, Variance, Covariance, Bias
\newcommand{\E}{\mathrm{E}}
\newcommand{\Var}{\mathrm{Var}}
\newcommand{\Cov}{\mathrm{Cov}}
\newcommand{\Bias}{\mathrm{Bias}}

\begin{document}

\maketitle

\pagebreak

\section*{Problem 4.43}
	\begin{enumerate}[(a)]
		\item $A = \{1, 2\}, B = \{1, 4\}, C = \{1, 5\}$
		\item $A = \{1\}, B = \{2\}, C = \{1, 2\}$
		\item
		
		Let $x \in B$. Two cases arise:
		
		\subsection*{Case 1: $x \in A$}
			Then $x \in A \cap B \implies x \in A \cap C \implies x \in A$ and $x \in C \implies x \in C$.
		
		\subsection*{Case 2: $x \notin A$}
			Then $x \in A \cup B \implies x \in A \cup C \implies x \in A$ or $x \in C \implies x \in C$.
	
		Therefore, $B \subseteq C$.
		
		Similarly, WLOG, let $x \in C$. Therefore, $C \subseteq B$.
		
		Therefore, $B = C$.
	\end{enumerate}

\section*{Problem 4.48}
	$A = \{n \in \mathbb{Z} : 2 | n \}, B = \{n \in \mathbb{Z} : 4 | n\}$
	
	\subsection*{Part 1: If $n \neq 2k$ for some odd integer k, then $n \notin A-B$}
		Case 1: $n = 2k+1$ for some $k \in \mathbb{Z}$.
		
		Then, $n \notin A \implies n \notin A-B$

		Case 2: $n = 2k$ for some even $k \in \mathbb{Z}$.
		
		Then, $k = 2l$ and $n = 4l$ for some $l \in \mathbb{Z}$.
		
		Therefore, $n \in A$ and $n \in B \implies n \notin A-B$.
		
	\subsection*{Part 2: If $n = 2k$ for some odd k, then $n \in A-B$}
		$k = 2l+1$ for some $l \in \mathbb{Z}$.
		Therefore, $n = 2(2l+1) \implies n = 4l+2$
		
		Since $2 | 2(2l+1)$, $2 | n \implies n \in A$ and $4 \not| 4l+2 \implies n \notin B$. Therefore, $n \notin A-B$.

\section*{Problem 4.56}
	Let $x \in (A-B) \cup (A-C)$.
	
	WLOG, let $x \in (A-B)$. $x \in (A-B) \implies x \in A$ and $x \notin B \implies x \notin B \cap C \implies x \in A - (B \cap C)$.
	
	Therefore, $(A-B) \cup (A-C) \subseteq A - (B \cap C)$.
	
	Let $x \in A - (B \cap C)$. Therefore, $x \in A$ and $x \notin B \cap C$.
	
	WLOG, let $x \notin B$. $x \notin B \implies x \in A-B \implies x \in (A-B) \cup (A-C)$.
	
	Therefore, $A - (B \cap C) \subseteq (A-B) \cup (A - C)$.
	
	Therefore, $A - (B \cap C) = (A-B) \cup (A - C)$.
	
\section*{Problem 5.4}
	Let $n = 2$. $\frac{n(n+1)}{2} = 3$ and $\frac{(n+1)(n+2)}{2} = 6$. $3$ is odd, but $6$ is not odd. Hence, the implication is false.

\section*{Problem 5.15}
	Let $a$ and $b$ be odd integers such that $4 | a^2+b^2$.
	
	Let $a^2+b^2 = 4k_1, a = 2k_2+1, b = 2k_3+1$ for some $k_1,k_2,k_3 \in \mathbb{Z}$.
	
	$(2k_2+1)^2+(2k_3+1)^2 = 4k_1 \implies 4(k_2^2+k_2+k_3^2+k_3) + 2 = 4k_1 \implies 2 = 4(k_1-k_2^2-k_2+k_3^2+k_3)$.
	
	Since $k_1-k_2^2-k_2+k_3^2+k_3 \in \mathbb{Z}$, we proved $4 | 2$, which is a contradiction.
	
\section*{Problem 5.18}
	Let the irrational number be $k_1$ and the rational number be $\frac{p}{q}$ such that $q \neq 0$.
	
	Since $\frac{p}{q} \neq 0 \implies p \neq 0$.
	
	Let product $k_2$ be a rational number.
	
	Then, $k_2 = \frac{k_1p}{q}$. Since $p \neq 0$, $k_1 = \frac{k_2q}{p}$. Since $\frac{k_2q}{p}$ is rational, we have shown that $k_1$ is rational, which is a contradiction.

\section*{Problem 5.20}
	Let both $ar+s$ and $ar-s$ be rational.
	
	Then, $ar+s = \frac{k_1}{k_2}$ for $k_2 \neq 0$ and $ar-s = \frac{k_3}{k_4}$ for $k_4 \neq 0$.
	
	Adding, $2ar = \frac{k_1}{k_2}+\frac{k_3}{k_4} \implies a = \frac{k_1k_4+k_2k_3}{2rk_2k_4}$. Thus, we have shown that $a$ is rational, which is a contradiction.
	
\section*{Problem 5.22}
	Let $\sqrt{2}+\sqrt{3}$ be rational.
	
	Then, $\sqrt{2} + \sqrt{3} = \frac{p}{q}$ for $q \neq 0$. Further, since $\sqrt{2} + \sqrt{3} > 0, p \neq 0$.

\begin{eqnarray*}
	2 = (\frac{p}{q} - \sqrt{3})^2 \\
	2 = \frac{p^2}{q^2} + 3 - \frac{2\sqrt{3}p}{q} \\
	\frac{p^2}{q^2} + 1 = \frac{2\sqrt{3}p}{q} \\
	\sqrt{3} = \frac{p^2+1}{2pq}, 2pq \neq 0 \\
\end{eqnarray*}

By the result of exercise 5.21, we know that $\sqrt{3}$ is irrational. However, we have shown that $\sqrt{3}$ is rational, which is a contradiction.

\section*{Problem 5.34}
	Let $n$ and $m \in \mathbb{Z^+}$ such that $m^2+m+1=n^2$.
	
	Since for any positive integer $a$, $\sqrt{a} < a$,  $\sqrt{m^2+m+1} < m^2+m+1 \implies n < m^2+m+1$.
	
	$n^2 - m^2 = m+1$. $m+1 > 0 \implies n^2-m^2 > 0 \implies n^2 > m^2 \implies n > m$.
	
	

\end{document}
