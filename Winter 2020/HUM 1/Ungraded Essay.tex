\documentclass[11pt]{article}

\usepackage{fancyhdr}
\usepackage{extramarks}
\usepackage{amsmath}
\usepackage{amsthm}
\usepackage{amsfonts}
\usepackage{tikz}
\usepackage[plain]{algorithm}
\usepackage{algpseudocode}
\usepackage[shortlabels]{enumitem}

\usetikzlibrary{automata,positioning}

%
% Basic Document Settings
%

\topmargin=-0.45in
\evensidemargin=0in
\oddsidemargin=0in
\textwidth=6.5in
\textheight=9.0in
\headsep=0.25in

\linespread{1.1}

\pagestyle{fancy}
\lhead{\hmwkAuthorName}
\rhead{\hmwkClass\ (\hmwkClassInstructor): \hmwkTitle}
\lfoot{\lastxmark}
\cfoot{\thepage}

\renewcommand\headrulewidth{0.4pt}
\renewcommand\footrulewidth{0.4pt}

\setlength\parindent{0pt}

%
% Homework Details
%   - Title
%   - Due date
%   - Class
%   - Section/Time
%   - Instructor
%   - Author
%

\newcommand{\hmwkTitle}{Ungraded Essay}
\newcommand{\hmwkClass}{HUM 1}
\newcommand{\hmwkClassInstructor}{Dr. Denise Demetriou}
\newcommand{\hmwkAuthorName}{\textbf{Nalin Bhardwaj}}
\newcommand{\hmwkAuthorID}{A16157819}
\newcommand{\hmwkAuthorEmail}{nibnalin@ucsd.edu}

%
% Title Page
%

\title{
    \vspace{2in}
    \textmd{\textbf{\hmwkClass:\ \hmwkTitle}}\\
    \vspace{0.1in}\large{\textit{\hmwkClassInstructor}}
    \vspace{3in}
}

\author{\hmwkAuthorName \\ \hmwkAuthorID \\ \hmwkAuthorEmail}
\date{}

\renewcommand{\part}[1]{\textbf{\large Part \Alph{partCounter}}\stepcounter{partCounter}\\}


\begin{document}

\maketitle

\pagebreak

\subsection*{Q. Telemachus changes and matures in the course of his journey.  To what extent, if any, does he resemble his father (Odysseus) by the end of his journey, with respect to his abilities as a leader?}

Odysseus, the legendary king of Ithaca, makes a hard choice when he leaves for the battle of Troy: he chooses the glory and gore of war over peace and fulfilment for his kingdom, people, and most importantly, his own family. With grieving Laertes retiring to the countryside, this decision’s consequences, are, perhaps, felt most strongly by Telemachus, who is left growing up without a father figure. Where Odysseus is remembered as a “boy wonder” single handedly killing boars (Homer, {\it Odyssey}, 19.385-476), Telemachus suffers a tormented childhood, with suitors pestering his mother and only tall tales of a father to live by. The effect of this upbringing is particularly evident during Telemachus’s conversation with Athena, during which he expresses doubts about his relationship with Odysseus (Homer, 1.212-220). On face value, one might compare Odysseus and Telemachus and see stark differences in their ideology and expression. Odysseus is a composed, careful thinker whose actions hold clear intention, whereas Telemachus wears his emotions on his sleeves, and is often unclear about his aims. While Odysseus and Telemachus have had very different upbringings and have very different personalities, they share the most important qualities of a leader: courage, insight and strategy, and Homer is using Athena as a narrative technique to make the reader observe this. \\

Athena is one of the most important characters of the Odyssey, someone responsible for completely changing the tide of the epic on multiple instances. However, her actions seem quite suspicious and uncharacteristic for a goddess. Where most Homeric gods are self-motivated, emotional entities wielding unhealthy amounts of power, Athena seems to exist only to aid the journey of Odysseus and Telemachus. She uses her mystical powers very sparingly, mostly just acting as an “emotional support goddess” comforting and inspiring our heroes. I think it is not a coincidence that the goddess associated with wisdom, courage, and strategy lurks so close to our heroes. Homer does not use Athena as a character that drives the narrative using her godly powers. He uses her as a proxy for displaying our heroes’ inner strength and courage. Let us look at some of the instances where Athena’s aid seems critical to the epic: \\

When Telemachus is assembling a crew for his expedition to Sparta, Athena, disguised as Telemachus, is the one who finds the crew for his ship (Homer, 3.380-400). Perhaps, Telemachus {\it was} indeed the one who collected the crew, but, almost like a multiple personality disorder, he attributed his courage to gods, and in doing so, Homer helped us see that he is breaking his mould and discovering his own identity as a leader. Similarly, when Odysseus has to reach the palace in Phaeacia without being discovered, Athena “made him invisible with magic mist” (Homer, 7.41) and he slips through the town unnoticed. I think Odysseus is simply imagining this invisibility, and Homer is trying to show us a leader’s determination and disregard for the thoughts of those around him. It is also notable that the only instance where Athena’s interference is felt by someone other than the two heroes is during the feast hosted by King Nestor (Homer, 3.371-384). Perhaps, having lost their senses to wine, what the court men saw wasn’t Athena, but a glimpse of Telemachus’s greatness as a leader. \\

The metaphor of Athena is what separates leaders and heroes from mere mortals, an unwavering belief in one’s own goals and the ability to perform superhuman feats in the face of adversity. \\

While it is true that Telemachus and Odysseus couldn’t be further apart from each other in their personalities, they share the true skills of a leader: courage, insight and creativity, traits displayed as a function of Athena’s actions. Homer beautifully uses a character to display the underlying similarities of a father and son, even when they have had very different foundations: Odysseus, a hubris filled warrior who has achieved superhuman feats since childhood, and Telemachus, a tormented figure full of self doubt who got fed up of his circumstances, and finally took ownership of his own fate. \\


\end{document}
