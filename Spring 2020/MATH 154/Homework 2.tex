\documentclass{article}

\usepackage{fancyhdr}
\usepackage{extramarks}
\usepackage{amsmath}
\usepackage{amsthm}
\usepackage{amssymb}
\usepackage{amsfonts}
\usepackage{tikz}
\usepackage[plain]{algorithm}
\usepackage{algpseudocode}
\usepackage[shortlabels]{enumitem}
\usepackage{gensymb}
\usepackage{booktabs}
\usepackage{graphicx}
\usepackage{float}

\usetikzlibrary{automata,positioning}

%
% Basic Document Settings
%

\topmargin=-0.45in
\evensidemargin=0in
\oddsidemargin=0in
\textwidth=6.5in
\textheight=9.0in
\headsep=0.25in

\linespread{1.1}

\pagestyle{fancy}
\lhead{\hmwkAuthorName}
\rhead{\hmwkClass\ (\hmwkClassInstructor): \hmwkTitle}
\lfoot{\lastxmark}
\cfoot{\thepage}

\renewcommand\headrulewidth{0.4pt}
\renewcommand\footrulewidth{0.4pt}

\setlength\parindent{0pt}

%
% Homework Details
%   - Title
%   - Due date
%   - Class
%   - Section/Time
%   - Instructor
%   - Author
%

\newcommand{\hmwkTitle}{Homework\ 2}
\newcommand{\hmwkClass}{MATH 154}
\newcommand{\hmwkClassInstructor}{Professor Kane}
\newcommand{\hmwkAuthorName}{\textbf{Nalin Bhardwaj}}
\newcommand{\hmwkAuthorID}{A16157819}
\newcommand{\hmwkAuthorEmail}{nibnalin@ucsd.edu}

%
% Title Page
%

\title{
    \vspace{2in}
    \textmd{\textbf{\hmwkClass:\ \hmwkTitle}}\\
    \vspace{0.1in}\large{\textit{\hmwkClassInstructor}}
    \vspace{3in}
}

\author{\hmwkAuthorName \\ \hmwkAuthorID \\ \hmwkAuthorEmail}
\date{}

\renewcommand{\part}[1]{\textbf{\large Part \Alph{partCounter}}\stepcounter{partCounter}\\}

%
% Various Helper Commands
%

% Useful for algorithms
\newcommand{\alg}[1]{\textsc{\bfseries \footnotesize #1}}

\DeclareMathOperator*{\argmax}{arg\,max}
\DeclareMathOperator*{\argmin}{arg\,min}

% For derivatives
\newcommand{\deriv}[1]{\frac{\mathrm{d}}{\mathrm{d}x} (#1)}

% For partial derivatives
\newcommand{\pderiv}[2]{\frac{\partial}{\partial #1} (#2)}

% Integral dx
\newcommand{\dx}{\mathrm{d}x}

% Alias for the Solution section header
\newcommand{\solution}{\textbf{\large Solution}}

% Probability commands: Expectation, Variance, Covariance, Bias
\newcommand{\E}{\mathrm{E}}
\newcommand{\Var}{\mathrm{Var}}
\newcommand{\Cov}{\mathrm{Cov}}
\newcommand{\Bias}{\mathrm{Bias}}

\begin{document}

\maketitle

\pagebreak

\section*{Question 1}
	
	Let the graph be $G = (V, E)$. \\

	First, let us generate a partitioning into two sets from a spanning tree $T$. Let us root the tree arbitrarily at any node $root$, then assign $height(u) = $ length of path from $root$ to $u$ using only tree edges. Then, consider sets $A = \{ u \in V : height(u) \equiv 0 \pmod 2 \} $ and $B = \{ u \in V : height(u) \equiv 1 \pmod 2 \} $. Firstly, since $G$ is connected, $T$ must also be connected. Therefore, $A \cup B = V$. Secondly, since there is exactly one path between any pair of vertices in a tree, either $height(u) \equiv 0 \pmod 2$ or $height(u) \equiv 1 \pmod 2$. Therefore, $A \cap B = \phi$. Therefore, $\{A, B\}$ is a valid partition of $V$. \\
	
	Next, let us show that WLOG, there are no edge $(u, v)$ s.t. $u \in A$ and $v \in A$. Let $d(u, v)$ = length of path between $u$ and $v$ in $T$.
	
	\subsection*{Lemma 1.1: For all $u, v \in A$, $d(u, v) \equiv 0 \pmod 2$}
		Consider the path corresponding to $d(u, v)$ can be broken up as path from $u$ to $l$ followed by the path from $l$ to $v$, where $l$ is the lowest common ancestor of $u$ and $v$, i.e. $l = \argmax height(w) \forall w \in$ the path from $root$ to $u$ and the path from $root$ to $v$. Using this $l$, $d(u, v) = height(u) + height(v) - 2 \cdot height(l) \implies d(u, v) \equiv height(u) + height(v) - 2 \cdot height(l) \pmod 2 \implies d(u, v) \equiv height(u) + height(v) \pmod 2$. Since $height(u) \equiv height(v) \equiv 0 \pmod 2$, $d(u, v) \equiv 0 \pmod 2$. \\

	Assume, for the sake of contradiction, that there exists an edge $(u, v)$ s.t. $u \in A$ and $v \in A$. If $(u, v) \in T$, $d(u, v) = 1$ which contradicts Lemma 1.1. Alternately, if $(u, v) \notin T$, consider the cycle formed by the path in $T$ corresponding to $d(u, v)$ and the edge $(u, v) \notin T$. Since $d(u, v) \equiv 0 \pmod 2$, this cycle has odd length, which contradicts the fact that $G$ was a bipartite graph.
	
	Now, let us show that this partitioning is unique. WLOG, for any node $u \in A$, it must have neighbours $adj_u \subseteq \overline{A} = B$. Since $G$ is connected, $|adj_u| \geq 1$. If we arbitrarily fix $u \in A$, we fix the set each other node belongs to: If $d(u, v) \equiv 0 \pmod 2$, $v \in A$ and if $d(u, v) \equiv 1 \pmod 2$, $v \in B$. Therefore, the partitioning is unique.

\section*{Question 2}
	Let us arbitrarily label the nodes $1,2,3,4,5,6$. Let $deg(u)$ = degree of node $u$. For construction, let's fix the labels WLOG such that in the degree sequence, $\forall i,j$ s.t. $i < j$, $deg(i) \geq deg(j)$ in the isomorphic equivalent of the labelled tree we pick. Since we are generating a connected tree, number of edges = $5$ and we also know $1 \leq deg(v) \leq 5 \quad \forall v$. Now, let us use case bashing. $5$ possibilities arise for node $1$:
	
	\subsection*{$deg(1) = 5$}
		Then, since total number of edges = 5, all of which have node 1 as an endpoint, only one possibility arises:
		\begin{figure}[H]
			\centering
			\includegraphics[width=0.3\linewidth]{graphHW2-0.png}
		\end{figure}

	\subsection*{$deg(1) = 4$}
		Since 4 of the 5 edges have node 1 as an endpoint, the final edge must be connected to node 2 (since $deg(2) \geq deg(k) \quad \forall k > 2$
		\begin{figure}[H]
			\centering
			\includegraphics[width=0.3\linewidth]{graphHW2-1.png}
		\end{figure}
	\subsection*{$deg(1) = 3$}
		Then, consider $deg(2)$. If $deg(2) = 1$, $deg(k) = 1 \quad \forall k \geq 2 \implies \sum_{v} deg(v) = 3 + 5\cdot1 = 8 \neq 10 = 2\cdot|E|$, which violates handshake lemma. Therefore, $2 \leq deg(2) \leq 3$, i.e. 2 subcases arise for node 2:

			\subsubsection*{$deg(2) = 3$}
				In this case, $\sum_{k \geq 3} deg(k) = 10 - (deg(1) + deg(2)) = 10 - 6 = 4 \implies deg(k) = 1 \quad \forall k \geq 3$.
				\begin{figure}[H]
					\centering
					\includegraphics[width=0.3\linewidth]{graphHW2-2.png}
				\end{figure}

			\subsubsection*{$deg(2) = 2$}
				In this case, $\sum_{k \geq 3} deg(k) = 10 - (deg(1) + deg(2)) = 10 - 5 = 5 \implies deg(3) > 1$ (otherwise, $\sum_{k \geq 3} deg(k) = 4$). Therefore, $deg(3) = 2$. Since $deg(2) = deg(3)$, labels 2 and 3 are interchangeable. Note that either node 2 or node 3 must be connected to node 1 (Otherwise, since at least 3 nodes are connected to node 2 and 3, $deg(1) \neq 3$.) WLOG, let node 2 be the node connected to node 1.
				
				2 distinct cases arise based on location of node 3:
				\begin{figure}[H]
					\centering
					\includegraphics[width=0.3\linewidth]{graphHW2-3.png}
					\caption{Node 3 is connected to node 2}
				\end{figure}
				
				\begin{figure}[H]
					\centering
					\includegraphics[width=0.3\linewidth]{graphHW2-4.png}
					\caption{Node 3 is not connected to node 2}
				\end{figure}

	\subsection*{$deg(1) = 2$}
		In this case, since $deg(k) \leq 2 \quad \forall k$, the graph must be a chain.
		\begin{figure}[H]
			\centering
			\includegraphics[width=0.3\linewidth]{graphHW2-5.png}
		\end{figure}

	\subsection*{$deg(1) = 1$}
		This case is not possible, because if $deg(k) = 1 \quad \forall k$, $\sum_{k} deg(k) = 6 \neq 10 = 2\cdot|E|$, which contradicts the handshake lemma. \\
		
	Therefore, there exist 6 different non-isomorphic trees of 6 nodes, as drawn above.
	
\section*{Question 3}
	Let the graph $G = (V, E)$. Let the weight of an edge $e$ be denoted by $w_e$. WLOG, consider two vertices $u$ and $v$. Let the minimum weight edge between $u$ and $v$ in $T$ have weight $\alpha$ and connect $(x, y)$. Let the largest truck weight that can pass between two vertices $u$ and $v$ be $\beta$.
	
	Firstly, consider the path denoted by nodes $a_1, a_2, \cdots , a_k \in T$. Then, since, $w_{(a_i, a_{i+1})} \geq \alpha \quad \forall i < k$, using this path, a truck of weight at least $\alpha$ may pass between $u$ and $v$, i.e. $\beta \geq \alpha$.
	
	Next, assume, for the sake of contradiction, that $\beta > \alpha$, and the path a truck of weight $\beta$ can take be $b_1, b_2, \cdots, b_l$. Compare this path with the path $a_1, a_2, \cdots , a_k \in T$. For the minimum weight edge $(x, y)$ (having weight $\alpha$), there must be some $i < k$ for which $a_i = x$ and $a_{i+1} = y$. Corresponding to this, the path $b$ must \textbf{skip} this edge, and there must be some $p, q$ such that $b_p = a_r$ and $b_q = a_s$ but $b_{p+1} \neq a_{r+1}$ and $b_{q-1} \neq a_{s-1}$ for the largest $r \leq i$, and the smallest $s \geq i+1$. Informally, we pick the part of the path $b_p, \cdots, b_q$ that is used by trucks to skip over the minimum weight edge, using edges not in $T$. Now, Note that there are two paths between $a_r$ and $a_s$, $a_r, a_{r+1}, \cdots, a_s \in T$ and $b_p = a_r, b_{p+1}, \cdots, b_q = a_s \notin T$, i.e., the edge $(x, y)$ (with weight $\alpha$) is in a cycle. Further, all edges on $b$ have weight $\geq \beta$. Therefore, we can delete edge $(x, y)$ (with weight $\alpha$) and replace it with a new edge not in $T$ from the cycle(i.e., from the path $b$) to obtain a new spanning tree $T'$. Now, total weight of $T' \geq$ total weight of $T + \beta - \alpha \implies $ total weight of $T' >$ total weight of $T$. which contradicts our assumption that $T$ was the maximum spanning tree of $G$, and therefore, $\beta \leq \alpha$.
	
	Combining these results, $\beta = \alpha$, that is, the largest truck that can make it from $v$ to $u$ is always the same as the minimum edge weight on the unique path from $v$ to $u$ in $T$.
	
	
\section*{Question 4}
	$\sim 5$ hours.
\end{document}
