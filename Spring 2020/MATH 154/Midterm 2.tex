\documentclass{article}

\usepackage{fancyhdr}
\usepackage{extramarks}
\usepackage{amsmath}
\usepackage{amsthm}
\usepackage{amssymb}
\usepackage{amsfonts}
\usepackage{tikz}
\usepackage[plain]{algorithm}
\usepackage{algpseudocode}
\usepackage[shortlabels]{enumitem}
\usepackage{gensymb}
\usepackage{booktabs}
\usepackage{graphicx}
\usepackage{float}
\usepackage{seqsplit}

\usetikzlibrary{automata,positioning}

%
% Basic Document Settings
%

\topmargin=-0.45in
\evensidemargin=0in
\oddsidemargin=0in
\textwidth=6.5in
\textheight=9.0in
\headsep=0.25in

\linespread{1.1}

\pagestyle{fancy}
\lhead{\hmwkAuthorName}
\rhead{\hmwkClass\ (\hmwkClassInstructor): \hmwkTitle}
\lfoot{\lastxmark}
\cfoot{\thepage}

\renewcommand\headrulewidth{0.4pt}
\renewcommand\footrulewidth{0.4pt}

\setlength\parindent{0pt}

%
% Homework Details
%   - Title
%   - Due date
%   - Class
%   - Section/Time
%   - Instructor
%   - Author
%

\newcommand{\hmwkTitle}{Midterm\ 2}
\newcommand{\hmwkClass}{MATH 154}
\newcommand{\hmwkClassInstructor}{Professor Kane}
\newcommand{\hmwkAuthorName}{\textbf{Nalin Bhardwaj}}
\newcommand{\hmwkAuthorID}{A16157819}
\newcommand{\hmwkAuthorEmail}{nibnalin@ucsd.edu}

%
% Title Page
%

\title{
    \vspace{2in}
    \textmd{\textbf{\hmwkClass:\ \hmwkTitle}}\\
    \vspace{0.1in}\large{\textit{\hmwkClassInstructor}}
    \vspace{3in}
}

\author{\hmwkAuthorName \\ \hmwkAuthorID \\ \hmwkAuthorEmail}
\date{}

\renewcommand{\part}[1]{\textbf{\large Part \Alph{partCounter}}\stepcounter{partCounter}\\}

%
% Various Helper Commands
%

% Useful for algorithms
\newcommand{\alg}[1]{\textsc{\bfseries \footnotesize #1}}

\DeclareMathOperator*{\argmax}{arg\,max}
\DeclareMathOperator*{\argmin}{arg\,min}

% For derivatives
\newcommand{\deriv}[1]{\frac{\mathrm{d}}{\mathrm{d}x} (#1)}

% For partial derivatives
\newcommand{\pderiv}[2]{\frac{\partial}{\partial #1} (#2)}

% Integral dx
\newcommand{\dx}{\mathrm{d}x}

% Alias for the Solution section header
\newcommand{\solution}{\textbf{\large Solution}}

% Probability commands: Expectation, Variance, Covariance, Bias
\newcommand{\E}{\mathrm{E}}
\newcommand{\Var}{\mathrm{Var}}
\newcommand{\Cov}{\mathrm{Cov}}
\newcommand{\Bias}{\mathrm{Bias}}

\begin{document}

\maketitle

\pagebreak

\section*{Question 1}
	Let the number of vertices be $v$, edges be $e$ and faces be $f$. Further, let number of triangle faces be $x$ and square faces be $y$. Then, $x+y = f$. Then, for any vertex $a \in V$, since it is connected to exactly 5 faces, we have $deg(a) = 5$. Then, by the handshake lemma, $\sum_{a \in V} deg(v) = 2 \cdot e \implies e = \frac{5\cdot v}{2}$. Further, since triangles has 3 sides and a square has 4 sides, using the dual-handshake lemma, $\sum_{a \in F} sides(a) = 2 \cdot e \implies 3 \cdot x + 4 \cdot y = 5 \cdot v$. Now, since polyhedra are planar graphs, using euler's formula $v - e + f = 2 \implies v - \frac{5 \cdot v}{2} + f = 2 \implies f = x + y = 2 + \frac{3 \cdot v}{2}$. Plugging $v = 72$, we have a system of two equations, we solve for the 2 variables: $x+y = 110 \implies 3x+3y = 330$ and $3x+4y = 72*5 = 360$. Subtracting, $y = 30$. Plugging back in, $x = 110-30 = 80$. Therefore, the polyhedra has $80$ triangle faces and $30$ square faces.

\section*{Question 2}
	The set of nodes with color 1 is $\{A, E, F, H\}$.
	
	The set of nodes with color 2 is $\{B, C, D, G\}$.
	
	The set of nodes with color 3 is $\{I, J\}$.

\section*{Question 3}
		\begin{figure}[H]
			\centering
			\includegraphics[width=0.7\linewidth]{IMG_0484.jpg}
		\end{figure}

\section*{Question 4}
	Let $V_G$ represent the set of vertices in a graph $G$ and $E_G$ represent the set of edges in $G$, i.e. $G = (V_G, E_G)$. Also, for $u \in V_G$, let $color(u, G)$ = the color of node $u$ in a coloring corresponding to chromatic number of graph $G$.
	\subsection*{Lemma 1: For a complete graph $H$ and any graph $G'$, in any valid colouring of $H \cup G'$, for any $u, v \in V_H, color(u, H \cup G') \neq color(v, H \cup G')$. }
		This is because for any $u, v \in V_H$, since $H$ is a complete subgraph, there is an edge between vertex $u$ and $v$. Therefore, any valid coloring cannot use the same color for both the vertices. \\
		
	WLOG, let $\chi(H \cup E) \leq \chi(H \cup F)$.
	
	Consider the coloring of $H \cup E$ corresponding to $\chi(H \cup E)$. Using lemma 1, we know that for any $u, v \in V_H, color(u, H \cup E) \neq color(v, H \cup E)$. Therefore, each node $u \in V_H$ has a different color, i.e. $|\{color(u, H \cup E) \forall u \in V_H\}| = |V_H|$. WLOG, let the set of colors $\{color(u, H \cup E) \forall u \in V_H\}$ be the set $\{1, 2, \cdots , |V_H|\}$. This is always possible because we can just permute the set of colors so the labels/colors $\forall u \in V_H$ corresponds to $\{1, 2, \cdots , |V_H|\}$.
	
	Similarly, using the same reasoning as $H \cup E$, in the coloring of $H \cup F$, let the set of colors $\{color(u, H \cup F) \forall u \in V_H\}$ be the set $\{1, 2, \cdots , |V_H|\}$. Further, we can permute the set $\{color(u, H \cup F) \forall u \in V_H\}$ such that $color(u, H \cup E) = color(u, H \cup F) \quad \forall u \in V_H$. This is always possible because we can set the color of $u$ such that $color(u, H \cup E) = color(u, H \cup F)$ and permute the color of all vertices to obtain an equivalent valid coloring. 
	
	Now, using these two independent colorings, we can obtain a coloring of $G$ based on three possibilities $\forall u \in V_G$: If $u \in V_H$, then $color(u, G) = color(u, H \cup E) = color(u, H \cup F)$. Else if $u \in E$, $color(u, G) = color(u, H \cup E)$. Else, $u \in F$, and $color(u, G) = color(u, H \cup F)$. Since $E$ and $F$ are seperate components in $G \setminus H$, in this coloring, there are no edges $(u, v)$ such that $u \in E$ and $v \in F$. Som the set of neighbours for any node $u \in E \cup F$ is the same as the constituent colorings, where it had a valid color. For all nodes $u \in V_H$, since we obtained a coloring in both subgraphs $H \cup E$ and $H \cup F$ such that $u$ had the same color, it must be a valid color in $G = H \cup E \cup F$. So, we have a coloring of $G$ using $\chi(H \cup F)$ colors.
	
	We have thus shown $\chi(G) \leq \chi(H \cup F)$.
	
	Assume, for the sake of contradiction, that $\chi(G) < \chi(H \cup F)$. Then, consider the coloring of nodes in $G$ corresponding to $\chi(G)$. Since $H \cup F$ is an induced subgraph of $G$, any coloring valid in $G$ will also be valid in $H \cup F$. This is because the set of constraints defined by edges $E_{H \cup F} \subseteq E_{G}$. Therefore, we have obtained a coloring of $H \cup F$ such that we use fewer than $\chi(H \cup F)$ colors, which is a contradiction. Therefore, $\chi(G) \geq \chi(H \cup F)$.
	
	Combining results, $\chi(G) = \chi(H \cup F)$.

\section*{Question 5}
	I have permission to take the exam at a non-standard time.
	
	Start time: 7:30 AM Pacific Time. End time: 8:30 AM Pacific Time. Date: Saturday, May 23.
\end{document}
