\documentclass{tufte-book} % Use the tufte-book class which in turn uses the tufte-common class

\hypersetup{colorlinks} % Comment this line if you don't wish to have colored links

\usepackage{microtype} % Improves character and word spacing

\usepackage{lipsum} % Inserts dummy text

\usepackage{booktabs} % Better horizontal rules in tables

\usepackage{graphicx} % Needed to insert images into the document
\graphicspath{{graphics/}} % Sets the default location of pictures
\setkeys{Gin}{width=\linewidth,totalheight=\textheight,keepaspectratio} % Improves figure scaling

\usepackage{fancyvrb} % Allows customization of verbatim environments
\fvset{fontsize=\normalsize} % The font size of all verbatim text can be changed here

\newcommand{\hangp}[1]{\makebox[0pt][r]{(}#1\makebox[0pt][l]{)}} % New command to create parentheses around text in tables which take up no horizontal space - this improves column spacing
\newcommand{\hangstar}{\makebox[0pt][l]{*}} % New command to create asterisks in tables which take up no horizontal space - this improves column spacing

\usepackage{xspace} % Used for printing a trailing space better than using a tilde (~) using the \xspace command

\newcommand{\monthyear}{\ifcase\month\or January\or February\or March\or April\or May\or June\or July\or August\or September\or October\or November\or December\fi\space\number\year} % A command to print the current month and year

\newcommand{\openepigraph}[2]{ % This block sets up a command for printing an epigraph with 2 arguments - the quote and the author
\begin{fullwidth}
\sffamily\large
\begin{doublespace}
\noindent\allcaps{#1}\\ % The quote
\noindent\allcaps{#2} % The author
\end{doublespace}
\end{fullwidth}
}

\newcommand{\blankpage}{\newpage\hbox{}\thispagestyle{empty}\newpage} % Command to insert a blank page

\usepackage{makeidx} % Used to generate the index
\makeindex % Generate the index which is printed at the end of the document

%----------------------------------------------------------------------------------------
%	BOOK META-INFORMATION
%----------------------------------------------------------------------------------------

\title{Shadows of Stained Glass} % Title of the book

\author{Nalin Bhardwaj} % Author

\publisher{An Analytical Look at Horror Films} % Publisher

%----------------------------------------------------------------------------------------

\begin{document}

\frontmatter

%----------------------------------------------------------------------------------------

\maketitle % Print the title page

%----------------------------------------------------------------------------------------

\tableofcontents % Print the table of contents

\mainmatter

% Proposal

\chapter{Proposal}
\label{ch:0}

One of my favourite “things of beauty” are stained glass windows. By themselves, they contain intricate works of art with vivid colour palettes that overlay silhouettes of the outside world. They’re like a dreamy, imaginative capture of the beauty of nature and of the precise capability of human to weave intricacy into the commonplace object. But what make stained glass an extremely unique medium is its shadow. The shadows stained glass windows cast serve a utilitarian architectural purpose (that of lighting insides of churches and other buildings) while being a deep expression of colour and emotion. It is almost as if every stained glass panel has the capability to tell a multifaceted story by itself.

\begin{marginfigure}
\includegraphics[width=\linewidth]{graphics/cathedral.jpg}
\caption{The north rose window of the Chartres Cathedral}
\label{fig:marginfig}
\end{marginfigure}

\begin{marginfigure}
\includegraphics[width=\linewidth]{graphics/shadows.jpg}
\caption{Nasir-ol-Molk Mosque, Shiraz, Iran}
\label{fig:marginfig}
\end{marginfigure}

Films, as an art form, have the unique ability to fully engage our auditory and visual senses, just like the beautiful imagery of stained glass. Unfortunately, a side effect of the format of films is that our untrained brains are so engrossed in the momentary frames that they tend to miss the larger patterns of filmmaking. They neglect the “shadows” films cast. Most of us give very little thought to the underlying philosophy of cuts, narrative structures and sound design.

In my essay, I want to explore the “shadows” of these stained glass windows (i.e. films) to understand the patterns exhibited by the church (i.e the art of filmmaking). I will take an objective, data-backed approach to analysing films, using code to make observations about films. My project will be broadly categorised into three parts: the visual, the auditory and the written. In the visual part, I will look at the frames that compose films, analysing colour palettes and tones and looking at some of the psychological underpinnings of colour theory in film (and horror). The second part will look at sound as a complementary art form, focusing on different aspects of background music and sound effects as a tool that has the power to manipulate emotion. The final third part will bind my observations together. By looking at screenplays and their final products, I will analyse the manifestation of the uncanny in film: the medium that has made horror one of the most influential, accessible agents of expression.

\backmatter

%----------------------------------------------------------------------------------------
%	BIBLIOGRAPHY
%----------------------------------------------------------------------------------------

\nocite{*}

\bibliography{bibliography} % Use the bibliography.bib file for the bibliography
\bibliographystyle{plainnat} % Use the plainnat style of referencing

%----------------------------------------------------------------------------------------

\printindex % Print the index at the very end of the document

\end{document}